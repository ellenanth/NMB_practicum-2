In de functie naturalspline stellen we een natuurlijke kubische splinefunctie op op basis van een stel gegeven punten [x,f]. Hierbij evalueren we dit resultaat voor de gegeven waarden t. Dit gebeurt in verschillende stappen:
\begin{enumerate}
\item Stel een stelsel op met de tridiagonale co\"effici\"entenmatrix (5.19) en het rechterlid (5.20) uit het handboek.
\item Los dit stelsel op om de tweede afgeleiden in elke abscis te bepalen.
\item Bepaal de onbekende constanten $c_{1i}$ en $c_{2i}$ voor elke i met de formules (5.14) en (5.15) uit het handboek.
\item Met al deze gegevens y bepalen aan de hand van formule (5.13) uit het handboek.
\end{enumerate}
Het kan voorvallen dat de waarde van t buiten het interval van de gegeven abscissen ligt. Dit is niet ideaal, maar als dit voorvalt geven we y de functiewaarde die hoort bij de abscis op de rand. Ook als t juist op een abscis ligt, kunnen we rekenwerk besparen en geven we y de functiewaarde uit de gegeven punten.
\\
Om de correctheid van de implementatie te testen hebben we de functie sin(x) benaderd op het interval [-2,2] met equidistante punten. Hierbij gebruikten we 4 equidistante abscissen en 11 evaluatiepunten. Door dit zowel op papier als in matlab uit te rekenen kunnen stap per stap de waarden voor A, b, S, $c_1$, $c_2$ en y gecontroleerd worden.
\\
\lstinputlisting[
  style      = Matlab-editor,
  basicstyle = \mlttfamily,
]{naturalspline.m}

